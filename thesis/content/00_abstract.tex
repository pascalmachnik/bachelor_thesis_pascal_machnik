\thispagestyle{plain}

\section*{Abstract}
In this thesis, the rare baryonic decay $\Lambda_b^0 \to \Lambda^0 \mu^+ \mu^-$ is measured for the first time using Run 3 data collected by the LHCb experiment in 2024. The corresponding Run 2 dataset is also analyzed, providing a baseline to assess how the LHCb detector and trigger upgrades for Run 3 affect the analysis of rare baryonic decays. Because of its relatively long lifetime, the neutral $\Lambda^0$ hyperon can decay either inside or outside the Vertex Locator (VELO), resulting in long tracks, which include hits in the VELO, or downstream tracks, which do not. Signal candidates are classified as Long--Long (LL) or Downstream--Downstream (DD), depending on the track types of both $\Lambda^0$ decay products. To account for differences in reconstruction, LL and DD events are analyzed separately. 

The measured Run 3 signal yields are lower relative to Run 2, partly due to the lack of simulation calibration and selection optimization in the Run 3 study. The reduction is especially pronounced for DD events, likely reflecting the commissioning status of the Upstream Tracker during the Run 3 data taking.
%The relatively long lifetime of the neutral $\Lambda^0$ hyperon can cause its decay products to originate beyond the Vertex Locator (VELO), resulting in long tracks, which include hits in the VELO, or downstream tracks, which do not.
%This thesis presents the first measurement of the rare baryonic decay $\Lambda_b^0 \to \Lambda^0 \mu^+ \mu^-$ using Run 3 data collected by the LHCb data, while also studying the corresponding Run 2 dataset to assess the impact of Upgrade I on the analysis of rare baryonic decays.
%This thesis presents the first measurement of the rare baryonic decay $\Lambda_b^0 \to \Lambda^0 \mu^+ \mu^-$ using data collected by the LHCb experiment during Run 3 of the LHC, providing a complementary perspective to the mesonic sector in the study of $B$-anomalies.
%By comparing the yields obtained in both runs, this study assesses the impact of the Run 3 detector and trigger upgrades on the analysis of rare baryonic decays. 
%The signal yield in Run 3 is observed to be reduced compared to Run 2. This reduction is partly due to the absence of simulation calibration and offline selection optimization, which results in suboptimal selection performance. The reduction is particularly pronounced in the DD category, likely reflecting the impact of the Upstream Tracker, which was still in the installation and commissioning phase during Run 3 data taking.

\section*{Kurzfassung}
\begin{foreignlanguage}{german}
In dieser Arbeit wird der seltene baryonische Zerfall $\Lambda_b^0 \to \Lambda^0 \mu^+ \mu^-$ erstmals mit Run‑3-Daten des LHCb-Experiments aus dem Jahr 2024 gemessen. Die entsprechenden Run‑2-Daten werden ebenfalls analysiert und dienen als Referenz, um zu bewerten, wie sich die für Run 3 durchgeführten Detektor- und Trigger‑Upgrades am LHCb-Experiment auf die Analyse seltener baryonischer Zerfälle auswirken. Aufgrund seiner relativ langen Lebensdauer kann das neutrale $\Lambda^0$-Hyperon entweder innerhalb oder außerhalb des Vertex Locators (VELO) zerfallen. Dies führt zu long tracks, die Spuren im VELO hinterlassen, oder zu downstream tracks, bei denen diese Spuren ausbleiben. Signalkandidaten werden dementsprechend, abhängig von den Spurtypen beider $\Lambda^0$-Zerfallsprodukte, in Long-Long (LL)- oder Downstream-Downstream (DD)-Ereignisse klassifiziert. Um Unterschiede in der Rekonstruktion zu berücksichtigen, werden LL- und DD-Ereignisse separat analysiert.

Die in Run 3 gemessene Anzahl an Signalzerfällen fällt im Vergleich zu Run 2 geringer aus. Dieser Effekt ist zum Teil auf die bislang fehlende Simulationskalibrierung sowie die noch nicht optimierte Selektion der Run-3-Studie zurückzuführen. Besonders ausgeprägt ist die Reduktion bei DD-Ereignissen, was vermutlich den Inbetriebnahmestatus des Upstream-Trackers während der Datennahme in Run 3 widerspiegelt.
\end{foreignlanguage}


%Die relativ lange Lebensdauer des neutralen $\Lambda^0$-Hyperons kann dazu führen, dass seine Zerfallsprodukte außerhalb des Vertex Locators (VELO) entspringen. Dies resultiert entweder in  long tracks, die im VELO Spuren hinterlassen, oder in downstream tracks, bei denen diese Spuren ausbleiben.
%Diese Arbeit präsentiert die erste Run-3-Messung des seltenen baryonischen Zerfalls $\Lambda_b^0 \to \Lambda^0 \mu^+ \mu^-$ am LHCb-Experiment und eröffnet damit eine komplementäre Perspektive zum mesonischen Sektor bei der Untersuchung von $B$-Anomalien.Zusätzlich werden Daten aus Run 2 herangezogen, um einen direkten Vergleich zwischen den beiden Datennahmeperioden zu ermöglichen.

%In doing so, events are classified as Long--Long (LL) or Downstream-Downstream (DD), depending on the track types of both decay products of the $\Lambda^0$ hyperon.
