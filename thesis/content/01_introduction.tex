\chapter{Introduction}
\label{ch:introduction}
The Standard Model of particle physics (SM) \cite{salam1959,glashow1959,weinberg1967} is a very successful theory, confirmed by many experimental measurements. However, several observations, such as non-zero neutrino masses \cite{neutrino_osc}, are not yet explained within the SM, motivating the search for New Physics (NP). Rare decays, which are strongly suppressed in the SM, are highly sensitive to potential NP contributions. As a result, they provide tests of SM predictions and enable the indirect search for NP.

Measurements of mesonic and baryonic decays involving the Flavor-Changing Neutral Current transition $b \to s \, \ell^+ \, \ell^-$ show such deviations from SM predictions \cite{b-anomalies_1, b-anomalies_2}, known as $B$-anomalies, which hint at possible NP contributions. Precision measurements of rare decays involving beauty quarks are a key focus of the LHCb experiment. During the second long shutdown of the Large Hadron Collider (LHC), LHCb underwent a major upgrade, referred to as Upgrade I \cite{lhcb_upgrade_I}, which replaced several subdetectors and introduced a fully software-based trigger system with new readout electronics.

The rare decay $\Lambda_b^0 \to \Lambda^0 \mu^+ \mu^-$ serves as a representative of the $b \to s \, \ell^+ \ell^-$ transition in the baryonic sector, providing complementary information on $B$-anomalies to that obtained from the extensively studied mesonic sector. So far, no study has been performed on this decay using Run 3 data. The aim of this thesis is to measure the signal yields for this rare decay with data collected by the LHCb experiment during Run 2 and Run 3 of the LHC, enabling a comparison between the two data-taking periods. Because of its relatively long lifetime, the $\Lambda^0$ hyperon may decay either inside or beyond the Vertex Locator (VELO), resulting in two different track types for its decay products: long tracks, which include hits in the VELO, and downstream tracks reconstructed without VELO information. This thesis analyzes the two event types separately to account for differences in reconstruction, thereby providing a more detailed understanding of the Run 3 data quality.

An introduction to the SM and the decay $\Lambda_b^0 \to \Lambda^0 \mu^+ \mu^-$ is given in \cref{ch:theory}. \cref{ch:lhcb} then presents an overview of the LHCb detector and Upgrade I. The analysis strategies for Run 2 and 3 are detailed in \cref{ch:run2} and \cref{ch:run3}, respectively. They contain online, offline, and multivariate selections, as well as the extraction of signal yields via invariant mass fits. A comparison between the two data-taking periods is performed. Finally, \cref{ch:conclusion} summarizes the results.
