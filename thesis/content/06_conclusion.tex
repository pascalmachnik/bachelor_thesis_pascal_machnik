\chapter{Conclusion and Outlook}
\label{ch:conclusion}
This thesis presents the first study of the rare baryonic decay $\Lambda_b^0 \to \Lambda^0 \mu^+ \mu^-$ using Run 3 data collected by the LHCb experiment. A comparison with the corresponding Run 2 dataset is performed to assess the impact of Upgrade I on the analysis of rare baryonic decays. The data are classified into LL and DD categories, corresponding to events in which the $\Lambda^0$ decay products are reconstructed as either two long tracks or two downstream tracks, respectively. To account for differences in reconstruction, the analysis is performed separately for these two categories. In Run 2, the signal simulation is calibrated, including the L0 trigger efficiencies, the muon identification variable \texttt{ProbNNMu}, and the kinematics and lifetime of the $\Lambda_b^0$ baryon, whereas no such calibration is applied in Run 3. The offline selection is optimized for Run 2 and is applied unchanged in Run 3. Finally, the signal yields are extracted via fits to the invariant mass distribution of the $\Lambda_b^0$ baryon.
The Run 2 yields are scaled to account for differences in integrated luminosity, providing a baseline for comparison. A reduction is observed for the Run 3 yields, which can be partly attributed to the absence of simulation calibration and the lack of selection optimization in the Run 3 study. The decrease is particularly pronounced for DD events, which can be explained by the status of the UT during data taking, as it was still in the installation and commissioning phase. Given that the UT is needed for the reconstruction of DD events, its performance directly impacts the DD yield, whereas LL events are less affected, as their reconstruction employs additional VELO information. A cross-check using the $J/\psi$ mode supports this hypothesis: in Run 3, both the rare and $J/\psi$ modes exhibit more LL than DD events, whereas in Run 2, the opposite trend is observed. While this thesis provides a preliminary assessment of the impact of Upgrade I, a detailed investigation of the individual contributions to the total efficiency, such as reconstruction and selection efficiencies, will be essential to fully understand the challenges and to optimize the analysis of Run 3 data. Furthermore, the analysis procedure in Run 3 can be optimized, likely increasing measured yields and enabling a more accurate assessment of simulation accuracy. In Run 3, $\qty{3.33}{\femto\barn}^{-1}$ have been recorded in less than eight weeks, compared to four years needed for $\qty{6}{\femto\barn}^{-1}$ in Run 2. Given that Run 3 is ongoing with a fully operational UT and downstream reconstruction included in HLT1, this thesis represents the first step toward measuring the rare decay $\Lambda_b^0 \to \Lambda^0 \mu^+ \mu^-$ in Run 3, laying the groundwork for future measurements of rare baryonic decays.

%This particular decay is chosen because it provides a complementary perspective to the extensively studied mesonic sector in the investigation of $B$-anomalies.
%The Run 2 dataset, collected between 2015 and 2018, corresponds to an integrated luminosity of $\mathcal{L} = \qty{6}{\femto\barn}^{-1}$ at a center-of-mass energy of $\sqrt{s} = \qty{13}{\tera\electronvolt}$, while the Run 3 dataset, recorded in 2024, corresponds to $\mathcal{L} = \qty{3.33}{\femto\barn}^{-1}$ at $\sqrt{s} = \qty{13.6}{\tera\electronvolt}$. 
%The signal selection consists of online and offline selection criteria, followed by a MVA with BDTs.
%The extracted signal yields amount to $\num{297(19)}$ for the LL and $\num{409(23)}$ for the DD category in Run 2, compared with $\num{132(13)}$ signal events for LL and $\num{99(11)}$ for DD in Run 3. 